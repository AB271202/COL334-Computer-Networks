\section{Appendix: Preparatory Tasks}
Here, we provide information about the various tools available for network analysis

\subsection{\texttt{ifconfig/ipconfig}}
  This is used to find the following for the network interfaces on the computer:
    \begin{description}
        \item[\textit{IP address}] An IP (Internet Protocol) address is a numerical label assigned to each device connected to a computer network that uses the Internet Protocol for communication. It serves two main purposes: identifying the host or network interface and providing the location of the host in the network. IP addresses can be either IPv4 (32-bit) or IPv6 (128-bit) and are written in a dotted-decimal format (e.g., {\tt172.31.225.222} for IPv4 or {\tt fe80::215:5dff:feeb:19f7} for IPv6).
        \item[\textit{Gateway}] A gateway, often referred to as a default gateway, is a network device (usually a router) that serves as an access point to other networks. It acts as an intermediary between devices within a local network and devices on other networks, including the internet. When a device on a local network wants to communicate with a device on another network, it sends the data to the gateway, which then forwards it to the appropriate destination.
        \item[\textit{Network mask}] A network mask, also known as a subnet mask, is used in conjunction with an IP address to determine the network portion and the host portion of the address. It is a binary pattern of bits that help divide an IP address into a network address and a host address. The network mask is typically represented in decimal format as four octets (e.g., {\tt255.255.255.0} for IPv4). It is used in the process of subnetting to identify which part of the IP address identifies the network and which part identifies the individual host within that network.
        \item[\textit{Hardware address}] A hardware address, also known as a MAC (Media Access Control) address, is a unique identifier assigned to a network interface card (NIC) by its manufacturer. It is a 48-bit address expressed in hexadecimal format and is used to identify a specific device on a local network. Each NIC in the world has its own unique MAC address, allowing devices to communicate with each other at the data link layer of the networking model.
        \item[\textit{DNS server}] A DNS (Domain Name System) server translates human-readable domain names, like www.google.com, into IP addresses that machines can understand. When you enter a URL in a web browser or try to access any internet resource, your device sends a DNS query to a DNS server. The DNS server then looks up the corresponding IP address associated with the domain name and returns it to your device, allowing it to establish a connection to the desired resource.
    \end{description}
    
    Running \texttt{ifconfig} on our system connected to Wifi gives the following output:

        
    \begin{code}
    root@IdeapadAB:~# ifconfig
    eth0: flags=4163<UP,BROADCAST,RUNNING,MULTICAST>  mtu 1500
      inet 172.31.225.222  netmask 255.255.240.0  broadcast 172.31.239.255
      inet6 fe80::215:5dff:feeb:19f7  prefixlen 64  scopeid 0x20<link>
      ether 00:15:5d:eb:19:f7  txqueuelen 1000  (Ethernet)
      RX packets 149  bytes 20663 (20.6 KB)
      RX errors 0  dropped 0  overruns 0  frame 0
      TX packets 13  bytes 1006 (1.0 KB)
      TX errors 0  dropped 0 overruns 0  carrier 0  collisions 0

    lo: flags=73<UP,LOOPBACK,RUNNING>  mtu 65536
      inet 127.0.0.1  netmask 255.0.0.0
      inet6 ::1  prefixlen 128  scopeid 0x10<host>
      loop  txqueuelen 1000  (Local Loopback)
      RX packets 0  bytes 0 (0.0 B)
      RX errors 0  dropped 0  overruns 0  frame 0
      TX packets 0  bytes 0 (0.0 B)
      TX errors 0  dropped 0 overruns 0  carrier 0  collisions 0
    \end{code}

    And on running it on mobile hotspot, we get the following output:
    \begin{code}
    root@IdeapadAB:~# ifconfig
    eth0: flags=4163<UP,BROADCAST,RUNNING,MULTICAST>  mtu 1500
        inet 172.31.225.222  netmask 255.255.240.0  broadcast 172.31.239.255
        inet6 fe80::215:5dff:feeb:19f7  prefixlen 64  scopeid 0x20<link>
        ether 00:15:5d:eb:19:f7  txqueuelen 1000  (Ethernet)
        RX packets 1035  bytes 154375 (154.3 KB)
        RX errors 0  dropped 0  overruns 0  frame 0
        TX packets 103  bytes 8962 (8.9 KB)
        TX errors 0  dropped 0 overruns 0  carrier 0  collisions 0

    lo: flags=73<UP,LOOPBACK,RUNNING>  mtu 65536
        inet 127.0.0.1  netmask 255.0.0.0
        inet6 ::1  prefixlen 128  scopeid 0x10<host>
        loop  txqueuelen 1000  (Local Loopback)
        RX packets 0  bytes 0 (0.0 B)
        RX errors 0  dropped 0  overruns 0  frame 0
        TX packets 0  bytes 0 (0.0 B)
        TX errors 0  dropped 0 overruns 0  carrier 0  collisions 0
    \end{code}

    %%%%%%%%%%%%%%%%%%%%%%% Why is the ip address in both the same? %%%%%%%%%%%%%%%%%%%%%%%

    {\tt eth0} and {\tt lo} are two different network interfaces. {\tt eth0} is associated with the Ethernet connection and {\tt lo} is the loopback(localhost) interface.

    Here is a description of the various fields in the output:
    \begin{description}
        \item[\textit{flags}] A set of flags that indicate the status of the network interface. 
        \item[\textit{mtu}] The Maximum Transmission Unit (MTU) is the size of the largest packet that can be transmitted over the network interface without being fragmented. The MTU is typically measured in bytes and can range from 64 to 65535 bytes.
        \item[\textit{inet}] The IPv4 address assigned to the network interface.
        \item[\textit{netmask}] The subnet mask for the IPv4 address. It helps determine the network and host portions of the IP address.
        \item[\textit{broadcast}] The broadcast address for the network. It is used to send data to all devices on the local network.
        \item[\textit{inet6}] The IPv6 link-local address with a prefix length of 64 bits. IPv6 addresses are written in hexadecimal format and are longer than IPv4 addresses.
        \item[\textit{ether}] The unique hardware address (MAC address) of the network interface card.
        \item[\textit{txqueuelen}] The length of the transmit queue.
        \item[\textit{RX packets}] The number of received packets.
        \item[\textit{TX packets}] The number of transmitted packets.
        \item[\textit{RX errors}] The number of receive errors.
        \item[\textit{TX errors}] The number of transmit errors.
        \item[\textit{dropped}] The number of dropped packets due to errors.
        \item[\textit{overruns}] The number of packets that had data sent beyond their allowed length.
        \item[\textit{frame}] The number of packets with framing errors.
        \item[\textit{collisions}] The number of packet collisions (i.e., when two devices transmit data at the same time).
    \end{description}
    
    The IP address of the smartphone can be found by "Settings$\rightarrow$About phone$\rightarrow$Status$\rightarrow$IP address"

\subsection{\tt ping}
This is used to discover if a particular IP address is online or not. For example, in the following code we are pinging www.google.com with packets of size 10 bytes and varying the TTL. We observe that as the TTL decreases, the packet doesn't reach the destination. This is because the TTL is decremented by 1 at each hop and when it reaches 0, the packet is dropped and an ICMP error message is sent back to the source. The source then knows that the packet didn't reach the destination and hence the destination is not online.
\begin{code}
root@IdeapadAB:~# ping -c 3 -s 50 -t 10 www.google.com
PING www.google.com (142.250.195.4) 50(78) bytes of data.
58 bytes from del12s09-in-f4.1e100.net (142.250.195.4): icmp_seq=1 ttl=55 time=82.3 ms
58 bytes from del12s09-in-f4.1e100.net (142.250.195.4): icmp_seq=2 ttl=55 time=67.1 ms
58 bytes from del12s09-in-f4.1e100.net (142.250.195.4): icmp_seq=3 ttl=55 time=33.1 ms

--- www.google.com ping statistics ---
3 packets transmitted, 3 received, 0% packet loss, time 2004ms
rtt min/avg/max/mdev = 33.130/60.834/82.271/20.545 ms
root@IdeapadAB:~# ping -c 3 -s 50 -t 9 www.google.com
PING www.google.com (142.250.195.4) 50(78) bytes of data.
From 142.251.52.213 (142.251.52.213) icmp_seq=1 Time to live exceeded
From 142.251.52.213 (142.251.52.213) icmp_seq=2 Time to live exceeded
From 142.251.52.213 (142.251.52.213) icmp_seq=3 Time to live exceeded

--- www.google.com ping statistics ---
3 packets transmitted, 0 received, +3 errors, 100% packet loss, time 2299ms
pipe 2
\end{code}



\subsection{\tt traceroute}
This gives you the sequence of routers that a packet traverses to get to a particular destination.
\begin{code}
C:\Users\Anish>tracert iitd.ac.in

Tracing route to iitd.ac.in [103.27.9.24]
over a maximum of 30 hops:

1     3 ms     4 ms     3 ms  192.168.107.98
2    39 ms    29 ms    21 ms  10.50.97.29
3    54 ms    46 ms    23 ms  10.50.97.223
4    58 ms    25 ms    34 ms  10.50.97.77
5   190 ms    30 ms    46 ms  dsl-ncr-dynamic-017.24.23.125.airtelbroadband.in [125.23.24.17]
6    63 ms    37 ms    27 ms  116.119.109.76
7    51 ms    38 ms    26 ms  49.44.187.164
8     *        *        *     Request timed out.
9     *        *        *     Request timed out.
10    38 ms    27 ms    27 ms  136.232.148.178
11     *        *        *     Request timed out.
12     *        *        *     Request timed out.
13     *        *        *     Request timed out.
14    53 ms    36 ms    60 ms  103.27.9.24
15    85 ms    35 ms    36 ms  103.27.9.24
16   148 ms   101 ms    86 ms  103.27.9.24

Trace complete.
\end{code}
\subsection{\tt nslookup}
This command helps you communicate with DNS servers to get the IP address for a particular hostname.

\subsection{\tt nmap}
This is a handy network diagnostics tool that you can use to discover which hosts are online in the network, and even try to infer what operating system the hosts might be running.

\subsection{\tt wireshark}
This is a very useful tool to sniff packets on the wire (or wireless medium). Sniffed data is parsed by wireshark and presented in an easily readable format with details of the protocols being used at different layers.
